Tyrimo metu buvo įsigilinta į molekulinės dinamikos metodus ir jų praktinį taikymą.
Taip pat implementuotas vandens modelis paremtas prediktoriaus - korektoriaus interavimo metodu,
bet algoritmas nėra stabilus ir skaičiavimo trukmė trumpa.
Atlikus šį tyrimą galima daryti tokias išvadas:

\begin{itemize}
    \item Lennard-Jones potencialų skaičiavimo metodas labai jautrus \({\Delta}t\), dėl stiprios atstumimo jėgos peržengus ribą.
        Todėl negalima naudoti optimalesnių metodų su kintamu \({\Delta}t\), ar Runge-Kuta tipo metodų.

    \item Molekulėms su tvirtais ryšiais (kaip vandens TIP4P) reikia skaičiuoti kampinius greičius.
        Tam tinkamiausias yra prediktoriaus - korektoriaus integravimo metodas, nes yra tikslesnis palyginus su Leapfrog tipo metodais
        ir leidžia apskaičiuoti pozicijas ir greičius nenaudojant trigonometrinių funkcijų.
\end{itemize}
