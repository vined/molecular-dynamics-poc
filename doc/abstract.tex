Šiame darbe tiriama poliariarizuoto jėgos lauko modelio įtaka molekulinės dinamikos simuliacijose.
Bio molekulių modeliavime labiausiai paplitę supaprastinti jėgos laukai kurie naudoja Kulono dėsnį atomų elektrostatinės sąveikos įvertinimui.
Atomams yra priskiriamas vienodo stiprumo elektrostatinis krūvis, nepriklausomai nuo juos supančios aplinkos.
Naudojant tokius jėgos laukus galima atlikti simuliacijas su dideliu atomų skaičiumi, bet nukenčia tikslumas modeliuojant molekulių tarpusavio saveikas.
Tai aktualu didelių molekulių, kaip proteinai, lipidai, simuliacijose, kai modeliuojamas molekulės formos kitimas (folding).
Ir modeliuose, kur dipolinis momentas stipriai keičiasi, kaip vandens perėjimai tarp būsenų.
Poliarizuoti jėgos laukai sprendžia šią problemą.
Šiame darbe tirama fliuktuojančio krūvio metodo įtaka vandens molekulių klasterizavimosi modeliams.
