Šiame tyrime naudojamas fliuktuojančio krūvio jėgos laukas.
Skirtingai nuo kitų poliarizuotų jėgos laukų, fliuktuojantis krūvis gali būti taikomas keliems taškams (site) molekulėje, o ne kiekvienam atomui \cite{polar}.
Tai sumažina skaičiavimų kiekį. Be to skaičiavimai gali būti atlikti tame pačiame žingsnyje.

\begin{multline} \label{eq:fluct_charge}
U_{e}(r, q) = \sum\limits_{i=1}^{N_{molec}} \sum\limits_{\alpha=1}^{N_{site}}
{\bigg[\chi_{i\alpha}^0 q_{i\alpha} + \dfrac{1} {2} J_{i\alpha i\alpha}^0 q_{i\alpha}^2 \bigg]} \\
+ \sum\limits_{i=1}^{N_{molec}} \sum\limits_{\alpha=1}^{N_{site}} \sum\limits_{\beta>\alpha}^{N_{site}}
{J_{i\alpha i\beta} (r_{i\alpha i\beta}) q_{i\alpha} q_{i\beta}}
+ \sum\limits_{i=1}^{N_{molec}} \sum\limits_{j>i}^{N_{molec}} \sum\limits_{\alpha=1}^{N_{site}} \sum\limits_{\beta>\alpha}^{N_{site}}
{J_{i\alpha j\beta} (r_{i\alpha j\beta}) q_{i\alpha} q_{j\beta}}
\end{multline}

Formulė \ref{eq:fluct_charge} pakeičia anksčiau naudotą Kulono formulę \ref{eq:coloumb}.
Čia \(\chi_{ij}^0\) yra Mulikeno (Mulliken) elektroneigiamumas, \(J_{\alpha\alpha}\) - absoliutus kietumas (angl. \textit{absolute hardness}).
Iš formulės matosi, kad naudojamos tos pačios elektrostatinių krūvių reikšmės kaip ir Kulono dėsnyje,
bet jų stiprumas jau priklauso nuo aplink esančių krūvių.


%------------------------------------------------
%
%Tik du papildomi parametrai kiekvienam atomui.
%
%Over-polarization problem
%
%Charge transfers over atomic pairs
%
%ABEEM
