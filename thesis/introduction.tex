Molekulinė dinamika (MD) - tai sąveikų tarp atomų ir molekulių modeliavimo sritis.
Sąveikos jėgos yra apskaičiuojamos pagrinde remiantis Niutono klasikine mechanika ir termodinamikos dėsniu.
Modelį sudaro N atomų (arba iš jų sudarytos molekulės) iš tyrimų žinomoje arba atsitiktine tvarka parinktoje konfigūracijoje esančių ribotoje erdvėje.
Modeliavimas vyksta iteracijomis kurių laiko žingsniai yra labai trumpi - apie vieną femto sekundę.
Toks mažas žingsnis yra reikalingas, nes atomai juda labai greitai.
Visos simuliacijos trukmė skaičiuojama micro ar nano sekundėmis.
Ilgesnėse simuliacijose sunku užtikrinti modelio stabilumą.

Pagrindinį vaidmenį MD simuliacijoje atlieka atomo jėgos lauko modelis.
Atomai traukia kitus atomus Niutono dėsniais paremtomis jėgomis.
Bet nuo tam tikro atstumo dominuojančia jėga tampa Paulio draudimo principas, kuris neleidžia atomams kirstis.
Jėgos lauko modelio tikslas kuo geriau atkartoti šias sąveikas, bet taip pat, neapkrauti procesoriaus skaičiavimais, kurie turi mažai įtakos galutiniam rezultatui.
Vis tik nauji tyrimai rodo, kad standartiniai jėgos laukai yra pakankamai netikslūs ir tam tikrose situacijose jų rezultatai stipriai skiriasi nuo empirinių.

Pingant skaičiuojamajai galiai, galima naudoti tikslesnius modelius, kurie įtraukia daugiau sąveikų į skaičiavimus.
Poliarizuoti jėgos laukai yra vienas iš tokių tikslumo problemos sprendimo būdų.
Jie įvertina aplinkos įtaką atomo ar molekulės elektrostatiniam krūviui.
Nors ši sąveika yra dešimtis kartų mažesnė už atomų traukos ir stūmos jėgas, simuliacijos, įvertinančios atomų poliarizaciją yra tikslesnės.
Tai labiausiai aktualu modeliuose, kuriuose tarpusavyje sąveikauja molekulės arba didelę įtaką daro ne kovalentinis ryšys (pvz. vandenilinis ryšys).

Molekulinė dinamika taikoma cheminėje fizikoje, medžiagų tyrimuose ir bio-molekulių modeliavime vaistų pramonėje.

---------------------------------------------------------------

kokia motyvacija
kokie tikslai ir uzdaviniai
kas sukurta, kokios problemos isprestos, kokie tyrimai ivykdyti
kokie rezultatai pasiekti

trumpai aprašyta darbo struktūra

