Jėgos laukas yra pagrindinė modelio sudedamoji dalis.
Jis yra taikomas kiekvienam simuliacijos atomui daugelį kartų.
Nuo jo tikslumo priklauso galutinis rezultatas, bei skaičiavimo trukmė.


\begin{equation} \label{eq:upot}
    U(r) = U_b(r) + U_W(r) + U_e(r)
\end{equation}

%\begin{multline} \label{eq:upotdetailed}
%    U(r^N) = \sum\limits_{bonds}{\dfrac {k_i} {2} (l_i - l_{i,0})^2} + \sum\limits_{angles}{\dfrac {k_i} {2} (\theta_i - \theta_{i,0})^2} + \sum\limits_{torsions}{\dfrac {V_n} {2} (1 + cos(n\omega - \gamma))} \\
%  + \sum\limits_{i=1}^N \sum\limits_{j=i+1}^N {\bigg(4\varepsilon_{ij} \bigg[\bigg(\dfrac {\sigma_{ij}} {r_{ij}} \bigg)^{12} - \bigg(\dfrac {\sigma_{ij}} {r_ij} \bigg)^6 \bigg] + \dfrac {q_iq_j} {4\pi\sigma_0r_{ij}}} \bigg)
%\end{multline}

Pagrindinė potencinės energijos tarp dviejų sąveikaujančių dalelių formulė \ref{eq:upot} susidedada iš
\(U_b\) - ryšio su kitais molekulės atomais energijos, \(U_W\) - van der Valso jėgos ir \(U_e\) - elektrostatinės  energijos.
Čia \(r\) yra modelio atomų pozicijos.

\begin{equation} \label{eq:ubond}
U_b(r) = \sum\limits_{bonds}{\dfrac {k_i} {2} (l_i - l_{i,0})^2} + \sum\limits_{angles}{\dfrac {k_i} {2} (\theta_i - \theta_{i,0})^2} + \sum\limits_{torsions}{\dfrac {V_n} {2} (1 + cos(n\omega - \gamma))}
\end{equation}

Ryšio tarp susietų molėkulės atomų potencinės energijos formulė \ref{eq:ubond} įtraukia galimus ryšio laisvės laipsnius.


\begin{equation} \label{eq:uw}
U_W(r) = \sum\limits_{i=1}^N \sum\limits_{i\neq j} {U_{LJ}(r_{ij})}
\end{equation}

\begin{equation} \label{eq:lj}
    U_{LJ}(r_{ij}) = 4\varepsilon\Bigg[\bigg(\dfrac \sigma {r_{ij}}\bigg)^{12} - \bigg(\dfrac \sigma {r_{ij}}\bigg)^6\Bigg]
\end{equation}

Van der Valso jėgų įtaka dažniausiai skaičiuojama naudojant Lenard-Jones potencialų formulę \ref{eq:lj}.
Ji taip pat įtraukia Paulio draudimo principo įtaką \cite{molmod}.
Čia \(r_{ij}\) yra atstumas tarp jų.

\begin{equation} \label{eq:coloumb}
    U_{e}(r) = \sum\limits_{i=1}^N \sum\limits_{i\neq j} {\dfrac{q_i q_j}{r_{ij}}}
\end{equation}

\(U_e\) - elektrostatinių jėgų potencinė energija skaičiuojama remiantis Kulono formule \ref{eq:coloumb}.
Čia \(q_i\) ir \(q_j\) elekrostatiniai atomų krūviai.

MD simuliacijose dažniausiai naudojamas būtent šiomis formulėmis paremti jėgos laukai \cite{polar}.

