\subsection{Molekulinė dinamika}
\label{sec:molecular_dynamics}


Molekulinė dinamika - tai supaprastintas atomų ir molekulių tarpusavio sąveikos modelis.
Fiksuotas atomų skaičius sąveikauja uždarame tūryje reminatis klasikine Niutono mechanika,
nes ji labai gerai atitinka kvantinės mechanikos dėsnius modelio naudojamos temperatūros ir slėgio rėžiuose,
bet kartu yra greitesnė ir paprastesnė.
Laikoma, kad sistemos suminė energija simuliacijos metu nekinta.
Taip pat gali būti fiksuoti slėgis ir temperatūra \cite{hitch}.
Modeliuojamų atomų skaičius svyruoja nuo kelių šimtų iki keliolikos tūkstančių.
Kadangi modelis reikalauja daug skaičiavimo resursų (vienas žingsnis yra apie 6N operacijų, kur N yra atomų skaičius),
simuliacijos nėra ilgos ir žingsniai labai maži (apie vieną femto sekundę), nes su didesniais žingsniais algoritmas tampa nestabilus.


\subsection{Molekulinės dinamikos modelis}
\label{sec:molecular_dynamics_model}


MD simuliaciją sudaro trys dalys: sąveikos modelis (jėgos laukas), modelio sąlygos ir parametrai, pradinės pozicijos ir greičiai.

Pradinės pozicijos gali būti parinktos atsitiktine tvarka, bet tai nėra rekomenduotina, nes turi būti išlaikytas minimalus atstumas tarp dalelių.
Dažniausiai pozicijos parenkamos pseudo atsitiktine tvarka remiantis empiriniais eksperimentų duomenimis.
Pradinis greitis nėra toks svarbus, nes jis bus perskaičiuotas jau pirmame žingsnyje ir gali būti parinktas atsitiktine tvarka pagal Gauso skirstinį,
kad pradinis suminis sistemos momentas \(p\) būtų lygus nuliui.


\begin{figure}
\centering

\usetikzlibrary{shapes.geometric, arrows, arrows.meta, positioning, calc}
\tikzstyle{block} = [rectangle, draw=black, text width=8cm, text centered]
\tikzstyle{decision} = [diamond, draw=black, text width=1.7cm, text centered]
\tikzstyle{arrow} = [thick,->, >=stealth]

    \begin{tikzpicture}[node distance=1cm]
        \node (init) [block] {Pradinės padėtys ir greičiai};
        \node (newEnergies) [block, below of=init] {Skaičiuojamos jėgos};
        \node (update) [block, below of=newEnergies] {Skaičiuojamos naujos padėtys ir greičiai};
        \node (evaluation) [block, below of=update] {Iteracijos įvertinimas};
        \node (shouldEnd) [decision, below of=evaluation, yshift=-1.3cm] {Rezultatas pasiektas};
        \node (end) [block, below of=shouldEnd, yshift=-1.7cm] {Simuliacijos pabaiga};

        \draw [arrow] (init) -- (newEnergies);
        \draw [arrow] (newEnergies) -- (update);
        \draw [arrow] (update) -- (evaluation);
        \draw [arrow] (evaluation) -- (shouldEnd);
        \draw [arrow] (shouldEnd.east) -| node[pos=0.25, anchor=south] {ne} +(4,0) |- (newEnergies);
        \draw [arrow] (shouldEnd) -- node[anchor=east] {taip} (end);
    \end{tikzpicture}

\caption[Simuliacijos ciklas] {Simuliacijos ciklas}
\label{fig:steps}
\end{figure}

Diskrečiais laiko žingsniais atliekami žingzniai pavaizduoti iliustracijoje \ref{fig:steps}.
Po kiekvieno žingsnio suskaičiuojama suminė viso modelio energija.
Dalelėms susidūrus, dalis energijos virsta šiluma.
Todėl po kiekvieno ciklo taip pat skaičiuojamas sistemos slėgis ar temperatūra priklausomai nuo modelio tipo.
Skaičiavimai baigiasi kai pasiekiamas norimas rezultatas, pavyzdžiui dalelės pasiekė nustatytą klasterizavimosi lygį,
arba buvo atliktas norimas žingsnių kiekis.


\subsection{Ekvilibracija}
\label{sec:equilibration}
Dėl pradinių atsitiktinių parametrų dažniausiai sistema turi per didelį potencinės energijos kiekį.
Perteklinė potencinė energija laikui bėgant virsta kinetine, kuri kelia sistemos temperatūrą.
Norint sumažinti temperatūrą reikia sumažinti dalelių greičius.
Modeliavimo metu tai daroma keletą kartų kol sistema nusistovi ir pasiekamas norimas kinetinės energijos kiekis.
Tik tada prasideda simuliacija \(t = 0\)~\cite{ref_hitch}.


\subsection{Molekulinės dinamikos taikymai}
\label{sec:molecular_dynamics_applications}

Molekulinė dinamika taikoma naujų medžiagų tyrimuose, nano-technologijose, cheminėje fizikoje.

Taip pat dažnai naudojama biochemijoje proteinų ir makro molekulių simuliacijose.
Tai padeda geriau įvertinti vaistų veikliųjų medžiagų saveiką su tikslinėmis molekulėmis.
Inter-molekuliniai modeliai leidžia analizuoti ilgųjų molekulių kaip DNR sukinius (folding).

Molekulinė dinamika yra vienintelis būdas suprasti molekulių judėjimą, kai reikia suprasti procesus,
vykstančius mikroskopiniame lygyje, kurių neįmanoma ar per sudėtinga tiesiogiai stebėti~\cite{ref_art}.

