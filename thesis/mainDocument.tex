% Kompiuterijos katedros šablonas
% Template of Department of Computer Science II
% Versija 1.0 2015 m. kovas [ March, 2015]

\documentclass[a4paper,12pt,fleqn,tikz]{article}
\input{allPacks}

\newtoggle{inLithuanian}
%If the report is in Lithuanian, it is set to true; otherwise, change to false
\settoggle{inLithuanian}{true}

%create file preface.tex for the preface text
%if preface is needed set to true
\newtoggle{needPreface}
\settoggle{needPreface}{false}

\newtoggle{signaturesOnTitlePage}
\settoggle{signaturesOnTitlePage}{true}


\input{macros}

\begin{document}
    % #1 -report type, #2 - title, #3-7 students, #8 - supervisor
    \depttitlepage{Magistro tiriamasis darbas}{Fliuktuojančio krūvio jėgos lauko taikymas molekulinėje dinamikoje}
    {Edvinas Naraveckas}
    {}{}{}{}% students 2-5
    {dr. Tadas Meškauskas}

    \tableofcontents


    % Keywords and notations if needed
    \sectionWithoutNumber{Sutartinis terminų žodynas}{keywords}{
        MD - molekulinė dinamika (angl. \textit{Molecular dynamics})
    }

    % Both abstracts
    \bothabstracts{Šiame darbe tiriama poliariarizuoto jėgos lauko modelio įtaka molekulinės dinamikos simuliacijose.
Bio molekulių modeliavime labiausiai paplitę supaprastinti jėgos laukai kurie naudoja Kulono dėsnį atomų elektrostatinės sąveikos įvertinimui.
Atomams yra priskiriamas vienodo stiprumo elektrostatinis krūvis, nepriklausomai nuo juos supančios aplinkos.
Naudojant tokius jėgos laukus galima atlikti simuliacijas su dideliu atomų skaičiumi, bet nukenčia tikslumas modeliuojant molekulių tarpusavio saveikas.
Tai aktualu didelių molekulių, kaip proteinai, lipidai, simuliacijose, kai modeliuojamas molekulės formos kitimas (folding).
Ir modeliuose, kur dipolinis momentas stipriai keičiasi, kaip vandens perėjimai tarp būsenų.
Poliarizuoti jėgos laukai sprendžia šią problemą.
Šiame darbe tirama fliuktuojančio krūvio metodo įtaka molekulių klasterizavimosi modeliams.

}%tex-file of abstract in original language
    {Fluctuating Charge Force Field Application in Molecular Dynamics} %if work is in LT this title should be in English
    {In molecular dynamics simulations the algorithm work horses are force fields.
Most commonly simplified force fields are used.
They use Coulomb's law for electrostatic forces between atoms.
Atoms are assigned with electrostatic force value not accounting for their surrounding.
This helps to make calculations with large atoms count fast, but accuracy is noticebly diminished in cases there electrostatic energies play a bigger role.
Current studies of polarized force fields show increased accuracy in inter-molecular interactions models,
like water freezing or large molecules (proteins, lipids) folding.
One of them if fluctuating charge force field.
Electrostatic interactions are calculated between points on molecules called sites.
This allows to calculate charge in separate step, which eases force field incorporation to egzisting models.
As a bonus, sites can move along the bonds, which is proved to be happening in real life molecules.
This paper analyses fluctuating charge polarized force field application in molecules clumping models.
}%tex-file of abstract in other language


    % Introduction section: label is sec:intro
    \sectionWithoutNumber{\keyWordIntroduction}{intro}
    Molekulinė dinamika (MD) - tai sąveikų tarp atomų ir molekulių modeliavimo sritis.
Sąveikos jėgos yra apskaičiuojamos pagrinde remiantis Niutono klasikine mechanika ir termodinamikos dėsniu.
Modelį sudaro N atomų (arba iš jų sudarytos molekulės) iš tyrimų žinomoje arba atsitiktine tvarka parinktoje konfigūracijoje esančių ribotoje erdvėje.
Modeliavimas vyksta iteracijomis kurių laiko žingsniai yra labai trumpi - apie vieną femto sekundę.
Toks mažas žingsnis yra reikalingas, nes atomai juda labai greitai.
Visos simuliacijos trukmė skaičiuojama micro ar nano sekundėmis.
Ilgesnėse simuliacijose sunku užtikrinti modelio stabilumą.

Pagrindinį vaidmenį MD simuliacijoje atlieka atomo jėgos lauko modelis.
Atomai traukia kitus atomus Niutono dėsniais paremtomis jėgomis.
Bet nuo tam tikro atstumo dominuojančia jėga tampa Paulio draudimo principas, kuris neleidžia atomams kirstis.
Jėgos lauko modelio tikslas kuo geriau atkartoti šias sąveikas, bet taip pat, neapkrauti procesoriaus skaičiavimais, kurie turi mažai įtakos galutiniam rezultatui.
Vis tik nauji tyrimai rodo, kad standartiniai jėgos laukai yra pakankamai netikslūs ir tam tikrose situacijose jų rezultatai stipriai skiriasi nuo empirinių.

Pingant skaičiuojamajai galiai, galima naudoti tikslesnius modelius, kurie įtraukia daugiau sąveikų į skaičiavimus.
Poliarizuoti jėgos laukai yra vienas iš tokių tikslumo problemos sprendimo būdų.
Jie įvertina aplinkos įtaką atomo ar molekulės elektrostatiniam krūviui.
Nors ši sąveika yra dešimtis kartų mažesnė už atomų traukos ir stūmos jėgas, simuliacijos, įvertinančios atomų poliarizaciją yra tikslesnės.
Tai labiausiai aktualu modeliuose, kuriuose tarpusavyje sąveikauja molekulės arba didelę įtaką daro ne kovalentinis ryšys (pvz. vandenilinis ryšys).

Molekulinė dinamika taikoma cheminėje fizikoje, medžiagų tyrimuose ir bio-molekulių modeliavime vaistų pramonėje.

---------------------------------------------------------------

kokia motyvacija
kokie tikslai ir uzdaviniai
kas sukurta, kokios problemos isprestos, kokie tyrimai ivykdyti
kokie rezultatai pasiekti

trumpai aprašyta darbo struktūra





    % What is MD
    \newpage
    \section{Molekulinė dinamika ir jos taikymai}
    \label{sec:molecular_dynamics_section}
    \subsection{Molekulinė dinamika}
\label{sec:molecular_dynamics}


Molekulinė dinamika (MD) - tai supaprastintas atomų ir molekulių tarpusavio sąveikos modelis.
Fiksuotas atomų skaičius sąveikauja uždarame tūryje reminatis klasikine Niutono mechanika,
nes ji labai gerai atitinka kvantinės mechanikos dėsnius modelio naudojamos temperatūros ir slėgio rėžiuose,
bet kartu yra greitesnė ir paprastesnė.
Laikoma, kad sistemos suminė energija simuliacijos metu nekinta.
Taip pat gali būti fiksuoti slėgis ir temperatūra \cite{hitch}.
Modeliuojamų atomų skaičius svyruoja nuo kelių šimtų iki keliolikos tūkstančių.
Kadangi modelis reikalauja daug skaičiavimo resursų (vienas žingsnis yra apie 6N operacijų, kur N yra atomų skaičius),
simuliacijos nėra ilgos ir žingsniai labai maži (apie vieną femto sekundę), nes su didesniais žingsniais algoritmas tampa nestabilus.


\subsection{Molekulinės dinamikos modelis}
\label{sec:molecular_dynamics_model}


MD simuliaciją sudaro trys dalys: sąveikos modelis (jėgos laukas), modelio sąlygos ir parametrai, pradinės pozicijos ir greičiai.

Pradinės pozicijos gali būti parinktos atsitiktine tvarka, bet tai nėra rekomenduotina, nes turi būti išlaikytas minimalus atstumas tarp dalelių.
Dažniausiai pozicijos parenkamos pseudo atsitiktine tvarka remiantis empiriniais eksperimentų duomenimis.
Pradinis greitis nėra toks svarbus, nes jis bus normalizuotas jau pirmame žingsnyje ir gali būti parinktas atsitiktine tvarka pagal Gauso skirstinį,
kad pradinis suminis sistemos momentas \(p\) būtų lygus nuliui.
Tai reikalinga sistemos išlaikymui modeliuojamos erdvės centre.


\begin{figure}
\centering

\usetikzlibrary{shapes.geometric, arrows, arrows.meta, positioning, calc}
\tikzstyle{block} = [rectangle, draw=black, text width=8cm, text centered]
\tikzstyle{decision} = [diamond, draw=black, text width=1.7cm, text centered]
\tikzstyle{arrow} = [thick,->, >=stealth]

    \begin{tikzpicture}[node distance=1cm]
        \node (init) [block] {Pradinės padėtys ir greičiai};
        \node (newEnergies) [block, below of=init] {Skaičiuojamos jėgos};
        \node (update) [block, below of=newEnergies] {Skaičiuojamos naujos padėtys ir greičiai};
        \node (evaluation) [block, below of=update] {Iteracijos įvertinimas};
        \node (shouldEnd) [decision, below of=evaluation, yshift=-1.3cm] {Rezultatas pasiektas};
        \node (end) [block, below of=shouldEnd, yshift=-1.7cm] {Simuliacijos pabaiga};

        \draw [arrow] (init) -- (newEnergies);
        \draw [arrow] (newEnergies) -- (update);
        \draw [arrow] (update) -- (evaluation);
        \draw [arrow] (evaluation) -- (shouldEnd);
        \draw [arrow] (shouldEnd.east) -| node[pos=0.25, anchor=south] {ne} +(4,0) |- (newEnergies);
        \draw [arrow] (shouldEnd) -- node[anchor=east] {taip} (end);
    \end{tikzpicture}

\caption[Simuliacijos ciklas] {Simuliacijos ciklas}
\label{fig:steps}
\end{figure}

Diskretaus laiko tarpais atliekami žingsniai pavaizduoti iliustracijoje \ref{fig:steps}.
Po kiekvieno žingsnio suskaičiuojama suminė viso modelio energija.
Dalelėms susidūrus, dalis energijos virsta šiluma.
Todėl po kiekvieno ciklo taip pat skaičiuojamas sistemos temperatūra ir slėgis priklausomai nuo modelio tipo.
Skaičiavimai baigiasi kai pasiekiamas norimas rezultatas, pavyzdžiui dalelės pasiekė nustatytą klasterizavimosi lygį,
arba buvo atliktas norimas žingsnių kiekis.


\subsection{Ekvilibracija}
\label{sec:equilibration}
Dėl pradinių atsitiktinių parametrų dažniausiai sistema turi per didelį potencinės energijos kiekį.
Perteklinė potencinė energija laikui bėgant virsta kinetine, kuri kelia sistemos temperatūrą.
Norint sumažinti temperatūrą reikia sumažinti dalelių greičius.
Modeliavimo metu tai daroma keletą kartų kol sistema nusistovi ir pasiekamas norimas kinetinės energijos kiekis.
Tik tada prasideda simuliacija \(t = 0\)~\cite{hitch}.


\subsection{Molekulinės dinamikos taikymai}
\label{sec:molecular_dynamics_applications}

MD taikoma naujų medžiagų tyrimuose, nano-technologijose, cheminėje fizikoje.

Taip pat dažnai naudojama biochemijoje proteinų ir makro molekulių simuliacijose.
Tai padeda geriau įvertinti vaistų veikliųjų medžiagų saveiką su tikslinėmis molekulėmis.
Inter-molekuliniai modeliai leidžia analizuoti ilgųjų molekulių kaip DNR sukinius (folding).

Molekulinė dinamika yra vienintelis būdas pamatyti tikslų molekulių judėjimą ir kai reikia suprasti procesus,
vykstančius mikroskopiniame lygyje, kurių neįmanoma ar per sudėtinga tiesiogiai stebėti~\cite{art}.




    % Force field
    \newpage
    \section{Jėgos laukai}
    \label{sec:force_field}
    Jėgos laukas yra pagrindinė modelio sudedamoji dalis.
Jis yra taikomas kiekvienam simuliacijos atomui daugelį kartų.
Nuo jo tikslumo priklauso galutinis rezultatas, bei skaičiavimo trukmė.


\begin{equation} \label{eq:upot}
    U(r) = U_b(r) + U_W(r) + U_e(r)
\end{equation}

%\begin{multline} \label{eq:upotdetailed}
%    U(r^N) = \sum\limits_{bonds}{\dfrac {k_i} {2} (l_i - l_{i,0})^2} + \sum\limits_{angles}{\dfrac {k_i} {2} (\theta_i - \theta_{i,0})^2} + \sum\limits_{torsions}{\dfrac {V_n} {2} (1 + cos(n\omega - \gamma))} \\
%  + \sum\limits_{i=1}^N \sum\limits_{j=i+1}^N {\bigg(4\varepsilon_{ij} \bigg[\bigg(\dfrac {\sigma_{ij}} {r_{ij}} \bigg)^{12} - \bigg(\dfrac {\sigma_{ij}} {r_ij} \bigg)^6 \bigg] + \dfrac {q_iq_j} {4\pi\sigma_0r_{ij}}} \bigg)
%\end{multline}

Pagrindinė potencinės energijos tarp dviejų sąveikaujančių dalelių formulė \ref{eq:upot} susidedada iš
\(U_b\) - ryšio su kitais molekulės atomais energijos, \(U_W\) - van der Valso jėgos ir \(U_e\) - elektrostatinės  energijos.
Čia \(r\) yra modelio atomų pozicijos.

\begin{equation} \label{eq:ubond}
U_b(r) = \sum\limits_{bonds}{\dfrac {k_i} {2} (l_i - l_{i,0})^2} + \sum\limits_{angles}{\dfrac {k_i} {2} (\theta_i - \theta_{i,0})^2} + \sum\limits_{torsions}{\dfrac {V_n} {2} (1 + cos(n\omega - \gamma))}
\end{equation}

Ryšio tarp susietų molėkulės atomų potencinės energijos formulė \ref{eq:ubond} įtraukia galimus ryšio laisvės laipsnius.


\begin{equation} \label{eq:uw}
U_W(r) = \sum\limits_{i=1}^N \sum\limits_{i\neq j} {U_{LJ}(r_{ij})}
\end{equation}

\begin{equation} \label{eq:lj}
    U_{LJ}(r_{ij}) = 4\varepsilon\Bigg[\bigg(\dfrac \sigma {r_{ij}}\bigg)^{12} - \bigg(\dfrac \sigma {r_{ij}}\bigg)^6\Bigg]
\end{equation}

Van der Valso jėgų įtaka dažniausiai skaičiuojama naudojant Lenard-Jones potencialų formulę \ref{eq:lj}.
Ji taip pat įtraukia Paulio draudimo principo įtaką.
Čia \(r_{ij}\) yra atstumas tarp jų.

\begin{equation} \label{eq:coloumb}
    U_{e}(r) = \sum\limits_{i=1}^N \sum\limits_{i\neq j} {\dfrac{q_i q_j}{r_{ij}}}
\end{equation}

\(U_e\) - elektrostatinių jėgų potencinė energija skaičiuojama remiantis Kulono formule \ref{eq:coloumb}.
Čia \(q_i\) ir \(q_j\) elekrostatiniai atomų krūviai.

Molekulinės dinamikos simuliacijose dažniausiai naudojamas būtent šiomis formulėmis paremti jėgos laukai \cite{polar}.




    \subsection{Poliarizuoti jėgos laukai}
    \label{sec:polarized_force_fields}
    Dalyje \ref{ref_force_field} aprašytas jėgos laukas turi trūkumų.
Šis modelis neįvertina aplinkos įtakos polinės molekulės krūvio stiprumui, dėl ko gauti rezultatai gali nesutapti su empiriniais~\cite{ref_pol}.

Poliarizuoti jėgos laukai sprendžia šia problemą įtraukdami kintantį elektrostatinį krūvį į skaičiavimus.
Vienas iš šių metodų yra fliuktuojantis krūvio jėgos laukas.




    \subsubsection{Fliuktuojančio krūvio jėgos laukas}
    \label{sec:fluctuating_charge}
    Šiame tyrime naudojamas fliuktuojančio krūvio jėgos laukas.
Skirtingai nuo kitų poliarizuotų jėgos laukų, fliuktuojantis krūvis gali būti taikomas keliems taškams (site) molekulėje, o ne kiekvienam atomui \cite{polar}.
Tai sumažina skaičiavimų kiekį. Be to skaičiavimai gali būti atlikti tame pačiame žingsnyje.

\begin{multline} \label{eq:fluct_charge}
U_{e}(r, q) = \sum\limits_{i=1}^{N_{molec}} \sum\limits_{\alpha=1}^{N_{site}}
{\bigg[\chi_{i\alpha}^0 q_{i\alpha} + \dfrac{1} {2} J_{i\alpha i\alpha}^0 q_{i\alpha}^2 \bigg]} \\
+ \sum\limits_{i=1}^{N_{molec}} \sum\limits_{\alpha=1}^{N_{site}} \sum\limits_{\beta>\alpha}^{N_{site}}
{J_{i\alpha i\beta} (r_{i\alpha i\beta}) q_{i\alpha} q_{i\beta}}
+ \sum\limits_{i=1}^{N_{molec}} \sum\limits_{j>i}^{N_{molec}} \sum\limits_{\alpha=1}^{N_{site}} \sum\limits_{\beta>\alpha}^{N_{site}}
{J_{i\alpha j\beta} (r_{i\alpha j\beta}) q_{i\alpha} q_{j\beta}}
\end{multline}

Formulė \ref{eq:fluct_charge} pakeičia anksčiau naudotą Kulono formulę \ref{eq:coloumb}.
Čia \(\chi_{ij}^0\) yra Mulikeno (Mulliken) elektroneigiamumas, \(J_{\alpha\alpha}\) - absoliutus kietumas (angl. \textit{absolute hardness}).
Iš formulės matosi, kad naudojamos tos pačios elektrostatinių krūvių reikšmės kaip ir Kulono dėsnyje,
bet jų stiprumas jau priklauso nuo aplink esančių krūvių.


%------------------------------------------------
%
%Tik du papildomi parametrai kiekvienam atomui.
%
%Over-polarization problem
%
%Charge transfers over atomic pairs
%
%ABEEM




    % Algorithm
    \newpage
    \section{Implementacija}
    \label{sec:implementation}
    Algoritmas implementuotas C++ programavimo kalba.
Viso algoritmo su pagalbinėmis funkcijomis dydis - 950 eilučių (be tarpų ir komentarų).
Tokį programavimo kalbos pasirinkimą lėmė algoritmo poreikis skaičiavimo resursams.
C++ kaip ir C leidžia manipuliuoti atmintimi, bet kartu turi programavimą palengvinančių funkcijų, kaip vektoriai, stringai.
Algoritmo įgyvendinimui buvo naudojamos tik standartės C++ bibliotekos.

Rezultatų atvaizdavimui naudojamas grafikų piešimo įrankis GnuPlot.
Atomų judėjimo vizualizavimui naudojamas VMD įrankis.
Jis leidžia pažingsniui peržiūrėti atomų judėjimą, kurį nuskaito iš algoritmo išeksportuotų atomų pozicijų xyz formto failo.


\subsection{Prediktoriaus - korektoriaus metodas}
\label{sec:predictor}

Prediktoriaus - korektoriaus metodas - tai integravimo metodas, kuris remiasi skaičiavimais atliktais keliuose senesniuse žingsniuose.
Šiame darbe buvo implementuotas Adams-Bashforh metodo versija, kuri naudoja trijų prieš tai atliktų žingsnių duomenis.

\begin{equation} \label{eq:pred}
    P(x): x(t+h) = x(t) + hx'(t) +  h^2 \sum\limits_{i=1}^{k-1} {alpha_i f(t+[1-i]h)}
\end{equation}

\begin{equation} \label{eq:corr}
    C(x): x(t+h) = x(t) + hx'(t) +  h^2 \sum\limits_{i=1}^{k-1} {beta_i f(t+[2-i]h)}
\end{equation}


\subsection{Vandens modelis}
\label{sec:water}

\begin{figure}
    \centering
    \includegraphics[scale=0.65]{images/water.png}
    \caption{TIP4P konfigūracijos vandens molekulė}
    \label{fig:water}
\end{figure}

Modelyje naudojama TIP4P konfigūracijos vandens molekulė (\ref{fig:water} pav.).
Ji turi keturis sąveikos taškus.
Po vieną kiekviename atome bei vieną masės centre.
Šios molekulės konfigūracijos parametrai:

\begin{equation} \label{eq:water_conf}
    r_{OH} = 0.957 \si{\angstrom} \\
    r_{OM} = 0.15 \si{\angstrom} \\
    \angle HOH = 104.5 \si{\degree}
\end{equation}


\subsection{Pradinė sistemos būsena}
\label{sec:initial_state}

Visa simuliacijos erdvė užpildoma molekulėmis su vartotojo įvestu atstumu tarp jų.
Tuo met visų molekulių masės centrams pseudo-atsitiktiniu būdu yra priskiriamas pradinis greitis.
Tuo pačiu būdu priskiriamas kampinis greitis molekulės saveikų taškams.


\subsection{Saveikų }
\label{sec:}

--- r cut-off/shifted force potentials

--- Todo periodic boundaries

--- Ekvilibracija ir temperatura
Dažniausiai modeliuojamos uždaros sistemos kurios nepraranda dalelių ir šilumos. Temperatura:

\begin{equation}
    T(t) = \sum\limits_{i=1}^N {\dfrac {m_i} {N_{f}k_{b}} v_i^2}
\end{equation}


--- Gal dar apie rigid molekules

--- Kvaternijonai

--- Rotation matrix?

--- Hydrogen bond




    %Conclusions section
    \sectionWithoutNumber{\keyWordConclusions}{conclusions}
    Išvados

Svarbiausios darbo išvados, rekomendacijos vystymui ir pritaikymui


    %Conclusions section
    \sectionWithoutNumber{Ateities tyrimų planas}{futureWork}
    \begin{itemize}
    \item Pagerinti algoritmo stabilumą.
    \item Implementuoti geresnį vandenilinių ryšių įvertinimo metodą.
    \item Pritaikyti algoritmą paskirstytų skaičiavimų platformai.
    \item Pritaikyti algoritmą lanksčioms molekulėms.
\end{itemize}


    %file literatureSources.bib
    \referenceSources{literatureSources}



\end{document}
