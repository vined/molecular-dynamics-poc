Algoritmas implementuotas C++ programavimo kalba.
Viso algoritmo su pagalbinėmis funkcijomis dydis - 950 eilučių (be tarpų ir komentarų).
Tokį programavimo kalbos pasirinkimą lėmė algoritmo poreikis skaičiavimo resursams.
C++ kaip ir C leidžia manipuliuoti atmintimi, bet kartu turi programavimą palengvinančių funkcijų, kaip vektoriai, stringai.
Algoritmo įgyvendinimui buvo naudojamos tik standartės C++ bibliotekos.

Rezultatų atvaizdavimui naudojamas grafikų piešimo įrankis GnuPlot.
Atomų judėjimo vizualizavimui naudojamas VMD įrankis.
Jis leidžia pažingsniui peržiūrėti atomų judėjimą, kurį nuskaito iš algoritmo išeksportuotų atomų pozicijų xyz formto failo.


\subsection{Vandens modelis}
\label{sec:water}

\begin{figure}
    \centering
    \includegraphics[scale=0.65]{images/water.png}
    \caption{TIP4P konfigūracijos vandens molekulė}
    \label{fig:water}
\end{figure}

Modelyje naudojama TIP4P konfigūracijos vandens molekulė (\ref{fig:water} pav.).
Ji turi keturis sąveikos taškus.
Po vieną kiekviename atome bei vieną masės centre.
Šios molekulės konfigūracijos parametrai:

\begin{equation} \label{eq:water_conf}
r_{OH} = 0.957 \si{\angstrom} \\
r_{OM} = 0.15 \si{\angstrom} \\
\angle HOH = 104.5 \si{\degree}
\end{equation}


\subsection{Integravimo metodas}


\subsubsection{Prediktoriaus - korektoriaus metodas}
\label{sec:predictor}

Prediktoriaus - korektoriaus metodas - tai integravimo metodas, kuris remiasi skaičiavimais atliktais keliuose senesniuse žingsniuose.
Šiame darbe buvo implementuotas Adams-Bashforh metodo versija, kuri naudoja trijų prieš tai atliktų žingsnių duomenis.

\begin{equation} \label{eq:pred}
    P(x): x(t+h) = x(t) + hx'(t) +  h^2 \sum\limits_{i=1}^{k-1} {{\alpha}_i f(t+[1-i]h)}
\end{equation}

\begin{equation} \label{eq:corr}
    C(x): x(t+h) = x(t) + hx'(t) +  h^2 \sum\limits_{i=1}^{k-1} {{\beta}_i f(t+[2-i]h)}
\end{equation}


\subsubsection{Pradinė sistemos būsena}
\label{sec:initial_state}

Visa simuliacijos erdvė užpildoma molekulėmis su vartotojo įvestu atstumu tarp jų.
Tuo met visų molekulių masės centrams pseudo-atsitiktiniu būdu yra priskiriamas pradinis greitis.
Tuo pačiu būdu priskiriamas kampinis greitis molekulės saveikų taškams.


\subsubsection{Periodinės ribos}
\label{sec:periodic_boundaries}

Dažniausiai laikoma, kad modeliuojama sistema yra pakankamai toli nuo fizinių ribų kaip mėgintuvėlio sienelė ar skysčio paviršius.
Taip pat dėl ribotų skaičiavimo resusrsų laikoma modeliuojamas tūris yra labai nedidelis.
Kad simuliacija atitiktų realybę, naudojamos periodinių ribų sąlygą.
Kiekviena dalelė išėjusi iš modelio erdvės yra gražinama į ją kitoje pusėje.
Taip pat skaičiuojant sąveikas yra įvertinama ir kitoje erdvės pusėje esančių molekulių įtaka.
Naudojant periodines ribas erdvės plotis turėtų būti didesnis už dalelės saveikos atstumą,
kitu atveju dalelė pradės saveikauti pati su savimi.


\subsubsection{Kvaternijonai}
\label{sec:quaternions}

Implementacijoje sąveikos taškų pozicijų ir greičių skaičiavimams naudojami kvaternijonai.
Jie leidžia išvengti lėtesnių trigonometrinių funkcijų naudojimo.



\subsection{Matavimai}


\subsubsection{Sistemos temperatūra}
\label{sec:model_equilibration}

Dažniausiai modeliuojamos uždaros sistemos kurios nepraranda dalelių ir šilumos.
Tokiose sitemose reikia stebėti temperatūrą.
Simuliacijos pradžioje potencinės energijos perteklius greitai virsta kinetine energija ir kyla temperatūra.
Tokiu atveju taikoma ekvilibracija - sitemos temperatūros sumažinimas iki norimos, atitinkamai sumažinan visų dalelių kinetinę energiją.

\begin{equation}
    T(t) = \sum\limits_{i=1}^N {\dfrac {m_i} {N_{f}k_{b}} v_i^2}
\end{equation}


\subsubsection{Vandeniliai ryšiai}
\label{sec:hydrogen_bonds}

Pagrindinis matavimas, kuris leidžia spręsti apie poliarizuoto jėgos lauko taikymo efektyvumą,
tai vandenilinių ryšiu skaičius tam tikroje temperatūroje.
Pagal vartotojo įvestą maksimalią vandenilinio ryšio energiją, suskaičiuojami paskutiniu iteracijų ryšių skaičiaus vidurkiai.
Vidurkiai turėtų būti pasiskirstę Gauso skirstiniu aplink reikšmę 4.
Toks skaičius yra labiausiai tikėtinas, nes du vandens vandenilio atomai gali trauki kitų dviejų molekulių anglies atomus,
o anglies atomas yra pajėgus susisieti su dviem vandenilio atomais.


