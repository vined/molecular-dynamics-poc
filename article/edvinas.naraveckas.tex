% This is samplepaper.tex, a sample chapter demonstrating the
% LLNCS macro package for Springer Computer Science proceedings;
% Version 2.20 of 2017/10/04
%
\documentclass[runningheads]{llncs}
%
\usepackage{graphicx}
\usepackage[utf8x]{inputenc}
\usepackage[L7x]{fontenc}
\usepackage{amsmath}
\usepackage{tikz}

% Used for displaying a sample figure. If possible, figure files should
% be included in EPS format.
%
% If you use the hyperref package, please uncomment the following line
% to display URLs in blue roman font according to Springer's eBook style:
% \renewcommand\UrlFont{\color{blue}\rmfamily}

\begin{document}
%
\title{Poliarizuoto jėgos lauko taikymas molekulinėje dinamikoje}
%
%\titlerunning{Abbreviated paper title}
% If the paper title is too long for the running head, you can set
% an abbreviated paper title here
%
\author{Edvinas Naraveckas}
%
\authorrunning{E. Naraveckas}
% First names are abbreviated in the running head.
% If there are more than two authors, 'et al.' is used.
%
\institute{Vilnius University, Vilnius, Lithuania}
%
\maketitle              % typeset the header of the contribution
%
\begin{abstract}
Šiame darbe tiriama poliariarizuoto jėgos lauko modelio įtaka molekulinės dinamikos simuliacijose.
Bio molekulių modeliavime labiausiai paplitę supaprastinti jėgos laukai kurie naudoja Kulono dėsnį atomų elektrostatinės sąveikos įvertinimui.
Atomams yra priskiriamas vienodo stiprumo elektrostatinis krūvis, nepriklausomai nuo juos supančios aplinkos.
Naudojant tokius jėgos laukus galima atlikti simuliacijas su dideliu atomų skaičiumi, bet nukenčia tikslumas modeliuojant molekulių tarpusavio saveikas.
Tai aktualu didelių molekulių, kaip proteinai, lipidai, simuliacijose, kai modeliuojamas molekulės formos kitimas (folding).
Ir modeliuose, kur dipolinis momentas stipriai keičiasi, kaip vandens perėjimai tarp būsenų.
Poliarizuoti jėgos laukai sprendžia šią problemą.
Šiame darbe tirama fliuktuojančio krūvio metodo įtaka molekulių klasterizavimosi modeliams.

\keywords{Molekulinė dinamika  \and Poliarizuotas jėgos laukos \and Fliuktuojačio krūvio jėgos laukas.}
\end{abstract}
%
%
%
\section{Įvadas}

Molekulinė dinamika (MD) - tai sąveikų tarp atomų ir molekulių modeliavimo sritis.
Sąveikos jėgos yra apskaičiuojamos pagrinde remiantis Niutono klasikine mechanika ir termodinamikos dėsniu.
Modelį sudaro N atomų (arba iš jų sudarytos molekulės) iš tyrimų žinomoje arba atsitiktine tvarka parinktoje konfigūracijoje esančių ribotoje erdvėje.
Modeliavimas vyksta iteracijomis kurių laiko žingsniai yra labai trumpi - apie vieną femto sekundę
Toks mažas žingsnis yra reikalingas, nes atomai juda labai greitai.
Visos simuliacijos trukmė skaičiuojama micro ar nano sekundėmis.
Ilgesnėse simuliacijose sunku užtikrinti modelio stabilumą.

Pagrindinį vaidmenį MD simuliacijoje atlieka atomo jėgos lauko modelis.
Atomai traukia kitus atomus Niutono dėsniais paremtomis jėgomis.
Bet nuo tam tikro atstumo dominuojančia jėga tampa Paulio draudimo principas, kuris neleidžia atomams kirstis.
Jėgos lauko modelio tikslas kuo geriau atkartoti šias sąveikas, bet taip pat, neapkrauti procesoriaus skaičiavimais, kurie turi mažai įtakos galutiniam rezultatui.
Vis tik nauji tyrimai rodo, kad standartiniai jėgos laukai yra pakankamai netikslūs ir tam tikrose situacijose jų rezultatai stipriai skiriasi nuo empirinių.

Pingant skaičiuojamajai galiai, galima naudoti tikslesnius modelius, kurie įtraukia daugiau sąveikų į skaičiavimus.
Poliarizuoti jėgos laukai yra vienas iš tokių tikslumo problemos sprendimo būdų.
Jie įvertina aplinkos įtaką atomo ar molekulės elektrostatiniam krūviui.
Nors ši sąveika yra dešimtis kartų mažesnė už atomų traukos ir stūmos jėgas, simuliacijos, įvertinančios atomų poliarizaciją yra tikslesnės.
Tai labiausiai aktualu modeliuose, kuriuose tarpusavyje sąveikauja molekulės arba didelę įtaką daro ne kovalentinis ryšys (pvz. vandenilinis ryšys).


\section{Molekulinė dinamika ir jos taikymai}

\subsection{Molekulinė dinamika}

Molekulinė dinamika - tai supaprastintas atomų ir molekulių tarpusavio sąveikos modelis.
Fiksuotas atomų skaičius sąveikauja uždarame tūryje reminatis klasikine Niutono mechanika,
nes ji labai gerai atitinka kvantinės mechanikos dėsnius modelio naudojamos temperatūros ir slėgio rėžiuose,
bet kartu yra greitesnė ir paprastesnė.
Laikoma, kad sistemos suminė energija simuliacijos metu nekinta.
Taip pat gali būti fiksuoti slėgis ir temperatūra \cite{ref_molmod}.
Modeliuojamų atomų skaičius svyruoja nuo kelių šimtų iki keliolikos tūkstančių.
Kadangi modelis reikalauja daug skaičiavimo resursų (vienas žingsnis yra apie 6N operacijų, kur N yra atomų skaičius),
simuliacijos nėra ilgos ir žingsniai labai maži (apie vieną femto sekundę), nes su didesniais žingsniais algoritmas tampa nestabilus.


\subsection{Molekulinės dinamikos taikymai}

Molekulinė dinamika taikoma naujų medžiagų tyrimuose, nano-technologijose, cheminėje fizikoje.

Taip pat dažnai naudojama biochemijoje proteinų ir makro molekulių simuliacijose.
Tai padeda geriau įvertinti vaistų veikliųjų medžiagų saveiką su tikslinėmis molekulėmis.
Inter-molekuliniai modeliai leidžia analizuoti ilgųjų molekulių kaip DNR sukinius (folding).

Molekulinė dinamika yra vienintelis būdas suprasti molekulių judėjimą, kai reikia suprasti procesus,
vykstančius mikroskopiniame lygyje, kurių neįmanoma ar per sudėtinga tiesiogiai stebėti~\cite{ref_art}.


\section{Molekulinės dinamikos modelis}

MD simuliaciją sudaro trys dalys: sąveikos modelis (jėgos laukas), modelio sąlygos ir parametrai, pradinės pozicijos ir greičiai.

Pradinės pozicijos gali būti parinktos atsitiktine tvarka, bet tai nėra rekomenduotina, nes turi būti išlaikytas minimalus atstumas tarp dalelių.
Dažniausiai pozicijos parenkamos pseudo atsitiktine tvarka remiantis empiriniais eksperimentų duomenimis.
Pradinis greitis nėra toks svarbus, nes jis bus perskaičiuotas jau pirmame žingsnyje ir gali būti parinktas atsitiktine tvarka pagal Gauso skirstinį,
kad pradinis suminis sistemos momentas \(p\) būtų lygus nuliui.

\begin{figure}
    \centering

    \usetikzlibrary{shapes.geometric, arrows, arrows.meta, positioning, calc}
    \tikzstyle{block} = [rectangle, draw=black, text width=8cm, text centered]
    \tikzstyle{decision} = [diamond, draw=black, text width=1.7cm, text centered]
    \tikzstyle{arrow} = [thick,->, >=stealth]

    \begin{tikzpicture}[node distance=1cm]
        \node (init) [block] {Pradinės padėtys ir greičiai};
        \node (newEnergies) [block, below of=init] {Skaičiuojamos jėgos};
        \node (update) [block, below of=newEnergies] {Skaičiuojamos naujos padėtys ir greičiai};
        \node (evaluation) [block, below of=update] {Iteracijos įvertinimas};
        \node (shouldEnd) [decision, below of=evaluation, yshift=-1.3cm] {Rezultatas pasiektas};
        \node (end) [block, below of=shouldEnd, yshift=-1.7cm] {Simuliacijos pabaiga};

        \draw [arrow] (init) -- (newEnergies);
        \draw [arrow] (newEnergies) -- (update);
        \draw [arrow] (update) -- (evaluation);
        \draw [arrow] (evaluation) -- (shouldEnd);
        \draw [arrow] (shouldEnd.east) -| node[pos=0.25, anchor=south] {ne} +(4,0) |- (newEnergies);
        \draw [arrow] (shouldEnd) -- node[anchor=east] {taip} (end);
    \end{tikzpicture}

    \caption[Simuliacijos ciklas] {Simuliacijos ciklas}
    \label{fig:steps}
\end{figure}

Diskrečiais laiko tarpais atliekami žingsniai pavaizduoti iliustracijoje \ref{fig:steps}.
Po kiekvieno žingsnio suskaičiuojama suminė viso modelio energija.
Dalelėms susidūrus, dalis energijos virsta šiluma.
Todėl po kiekvieno ciklo taip pat skaičiuojamas sistemos slėgis ar temperatūra priklausomai nuo modelio tipo.
Skaičiavimai baigiasi kai pasiekiamas norimas rezultatas, pavyzdžiui dalelės pasiekė nustatytą klasterizavimosi lygį,
arba buvo atliktas norimas žingsnių kiekis.


\subsection{Ekvilibracija}

Dėl pradinių atsitiktinių parametrų dažniausiai sistema turi per didelį potencinės energijos kiekį.
Perteklinė potencinė energija laikui bėgant virsta kinetine, kuri kelia sistemos temperatūrą.
Norint sumažinti temperatūrą reikia sumažinti dalelių greičius.
Modeliavimo metu tai daroma keletą kartų kol sistema nusistovi ir pasiekamas norimas kinetinės energijos kiekis.
Tik tada prasideda simuliacija \(t = 0\)~\cite{ref_hitch}.


\subsection{Periodinių ribų sąlyga}

Retai aktualu kaip dalelės saveikauja su reliomis ribomis, kaip talpa kurioje jos yra, ar sąlytis su kita medžiaga.
Todėl dalelių sąveikos modeliuojamos uždaroje erdvėje kurią supa kita labai panaši erdvė.
Tai pasiekiama naudojan periodines ribas. T.y. dalelės, kurios išeina pro vieną modeliuojamos erdvės pusę,
yra grąžinamos atgal pro priešingą.
Taip pat, skaičiuojant saveikų jėgas yra įvertinama ir kitoje pusėje esančių dalelių įtaka.


\section{Jėgos laukas}
\label{ref_force_field}

Jėgos laukas yra pagrindinė modelio sudedamoji dalis.
Jis yra taikomas kiekvienam simuliacijoje dalyvaujančiam atomui.
Nuo jo tikslumo priklauso galutinis rezultatas, bei skaičiavimo trukmė.

\begin{equation} \label{eq:upot}
    U(r) = U_b(r) + U_W(r) + U_e(r)
\end{equation}

Jėgos laukas skaičiuoja kiekvieno atomo sitemoje potencinę energiją pagal formulę \ref{eq:upot}.
Čia \(r\) yra modelio atomų pozicijos, \(U_b\) - sąryšio potencinė energija.

\begin{equation}
    U_W(r) = \sum\limits_{i=1}^N \sum\limits_{i\neq j} {U_{LJ}(r_{ij})}
\end{equation}

\begin{equation} \label{eq:lj}
    U_{LJ}(r_{ij}) = 4\varepsilon\Bigg[\bigg(\dfrac \sigma {r_{ij}}\bigg)^{12} - \bigg(\dfrac \sigma {r_{ij}}\bigg)^6\Bigg]
\end{equation}

\(U_W\) - van der Valso jėgų įtaka, kuri dažniausiai skaičiuojama naudojant Lenard-Jones potencialų formulę \ref{eq:lj}.
Ji taip pat įtraukia Paulio draudimo principo įtaką.

\begin{equation} \label{eq:coloumb}
    U_{e}(r) = \sum\limits_{i=1}^N \sum\limits_{i\neq j} {\dfrac{q_i q_j}{r_{ij}}}
\end{equation}

\(U_e\) - elektrostatinių jėgų potencinė energija skaičiuojama remiantis Kulono formule \ref{eq:coloumb}.
Čia \(q_i\) ir \(q_j\) elekrostatiniai atomų krūviai, o \(r_{ij}\) yra atstumas tarp jų.

Molekulinės dinamikos simuliacijose dažniausiai naudojamas būtent šis algoritmas~\cite{polar}.


\subsection{Poliarizuoti jėgos laukai}

Dalyje \ref{ref_force_field} aprašytas jėgos laukas turi trūkumų.
Šis modelis neįvertina aplinkos įtakos polinės molekulės krūvio stiprumui, dėl ko gauti rezultatai gali nesutapti su empiriniais~\cite{polar}.

Poliarizuoti jėgos laukai sprendžia šia problemą įtraukdami kintantį elektrostatinį krūvį į skaičiavimus.
Vienas iš šių metodų yra fliuktuojantis krūvio jėgos laukas.


\subsection{Fliuktuojančio krūvio jėgos laukas}

Skirtingai nuo kitų poliarizuotų jėgos laukų fliuktuojantis krūvis taikomas keliems taškams molekulėje, o ne kiekvienam atomui.


Induced dipoles and Drude negali perkelti kruvio, tuo tarpu fliuktuojanti kruvis tą ir daro.

Tik du papildomi parametrai kiekvienam atomui.

r cut-off/shifted force potentials

Mulliken electro-negativity

Absolute hardness

Over-polarization problem

Charge transfers over atomic pairs

ABEEM
\section{Related work}
\section{Eksperimentas}
Papildomos programos: VMD ir gnuplot, dar gal xfig
\section{Išvados}

%
% ---- Bibliography ----
%
% BibTeX users should specify bibliography style 'splncs04'.
% References will then be sorted and formatted in the correct style.
%
% \bibliographystyle{splncs04}
% \bibliography{mybibliography}
%
\begin{thebibliography}{8}
    \bibitem{ref_art}
    Rapaport, D. C.: The art of molecular dynamics simulation, 2nd edn. Cambridge University Press (2004)

    \bibitem{ref_molmod}
    Leach, A. R.: Molecular modeling, principles and applications, 2nd edn. Prentice hall (2001)

    \bibitem{ref_hitch}
    Bopp, P. A., Hawlicka, E., Fritzsche, S.: The Hitchhiker’s guide to molecular dynamics. Springer International Publishing (2018)

    \bibitem{ref_pol}
    Lopes, P. E. M., Roux, B., MacKerell A. D.:
    Molecular modeling and dynamics studies with explicit inclusion of electronic polarizability: theory and applications Springer-Verlag 2009
\end{thebibliography}
\end{document}
